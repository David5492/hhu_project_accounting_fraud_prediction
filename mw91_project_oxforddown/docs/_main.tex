%%%%%%%%%%%%%%%%%%%%%%%%%%%%%%%%%%%%%%%%%%%%%%%%%%%%%%%%%%%%%%%
%% OXFORD THESIS TEMPLATE

% Use this template to produce a standard thesis that meets the Oxford University requirements for DPhil submission
%
% Originally by Keith A. Gillow (gillow@maths.ox.ac.uk), 1997
% Modified by Sam Evans (sam@samuelevansresearch.org), 2007
% Modified by John McManigle (john@oxfordechoes.com), 2015
% Modified by Ulrik Lyngs (ulrik.lyngs@cs.ox.ac.uk), 2018-, for use with R Markdown
%
% Ulrik Lyngs, 25 Nov 2018: Following John McManigle, broad permissions are granted to use, modify, and distribute this software
% as specified in the MIT License included in this distribution's LICENSE file.
%
% John commented this file extensively, so read through to see how to use the various options.  Remember that in LaTeX,
% any line starting with a % is NOT executed.  Several places below, you have a choice of which line to use
% out of multiple options (eg draft vs final, for PDF vs for binding, etc.)  When you pick one, add a % to the beginning of
% the lines you don't want.


%%%%% PAGE LAYOUT
% The most common choices should be below.  You can also do other things, like replacing "a4paper" with "letterpaper", etc.

% This one formats for two-sided binding (ie left and right pages have mirror margins; blank pages inserted where needed):
%\documentclass[a4paper,twoside]{templates/ociamthesis}
% This one formats for one-sided binding (ie left margin > right margin; no extra blank pages):
%\documentclass[a4paper]{ociamthesis}
% This one formats for PDF output (ie equal margins, no extra blank pages):
%\documentclass[a4paper,nobind]{templates/ociamthesis}

% As you can see from the uncommented line below, oxforddown template uses the a4paper size, 
% and passes in the binding option from the YAML header in index.Rmd:
\documentclass[a4paper, nobind]{templates/ociamthesis}


%%%%% ADDING LATEX PACKAGES
% add hyperref package with options from YAML %
\usepackage[pdfpagelabels]{hyperref}
% change the default coloring of links to something sensible
\usepackage{xcolor}

\definecolor{myurlcolor}{RGB}{0,0,139}
\definecolor{mycitecolor}{RGB}{0,33,71}

\hypersetup{
  hidelinks,
  colorlinks,
  linkcolor=.,
  urlcolor=myurlcolor,
  citecolor=mycitecolor
}



% add float package to allow manual control of figure positioning %
\usepackage{float}

% enable strikethrough
\usepackage[normalem]{ulem}

% use soul package for correction highlighting
\usepackage{color, soul}
\definecolor{correctioncolor}{HTML}{CCCCFF}
\sethlcolor{correctioncolor}
\newcommand{\ctext}[3][RGB]{%
  \begingroup
  \definecolor{hlcolor}{#1}{#2}\sethlcolor{hlcolor}%
  \hl{#3}%
  \endgroup
}
\soulregister\ref7
\soulregister\cite7
\soulregister\autocite7
\soulregister\textcite7
\soulregister\pageref7

%%%%% FIXING / ADDING THINGS THAT'S SPECIAL TO R MARKDOWN'S USE OF LATEX TEMPLATES
% pandoc puts lists in 'tightlist' command when no space between bullet points in Rmd file,
% so we add this command to the template
\providecommand{\tightlist}{%
  \setlength{\itemsep}{0pt}\setlength{\parskip}{0pt}}
 
% UL 1 Dec 2018, fix to include code in shaded environments
\usepackage{color}
\usepackage{fancyvrb}
\newcommand{\VerbBar}{|}
\newcommand{\VERB}{\Verb[commandchars=\\\{\}]}
\DefineVerbatimEnvironment{Highlighting}{Verbatim}{commandchars=\\\{\}}
% Add ',fontsize=\small' for more characters per line
\usepackage{framed}
\definecolor{shadecolor}{RGB}{248,248,248}
\newenvironment{Shaded}{\begin{snugshade}}{\end{snugshade}}
\newcommand{\AlertTok}[1]{\textcolor[rgb]{0.94,0.16,0.16}{#1}}
\newcommand{\AnnotationTok}[1]{\textcolor[rgb]{0.56,0.35,0.01}{\textbf{\textit{#1}}}}
\newcommand{\AttributeTok}[1]{\textcolor[rgb]{0.77,0.63,0.00}{#1}}
\newcommand{\BaseNTok}[1]{\textcolor[rgb]{0.00,0.00,0.81}{#1}}
\newcommand{\BuiltInTok}[1]{#1}
\newcommand{\CharTok}[1]{\textcolor[rgb]{0.31,0.60,0.02}{#1}}
\newcommand{\CommentTok}[1]{\textcolor[rgb]{0.56,0.35,0.01}{\textit{#1}}}
\newcommand{\CommentVarTok}[1]{\textcolor[rgb]{0.56,0.35,0.01}{\textbf{\textit{#1}}}}
\newcommand{\ConstantTok}[1]{\textcolor[rgb]{0.00,0.00,0.00}{#1}}
\newcommand{\ControlFlowTok}[1]{\textcolor[rgb]{0.13,0.29,0.53}{\textbf{#1}}}
\newcommand{\DataTypeTok}[1]{\textcolor[rgb]{0.13,0.29,0.53}{#1}}
\newcommand{\DecValTok}[1]{\textcolor[rgb]{0.00,0.00,0.81}{#1}}
\newcommand{\DocumentationTok}[1]{\textcolor[rgb]{0.56,0.35,0.01}{\textbf{\textit{#1}}}}
\newcommand{\ErrorTok}[1]{\textcolor[rgb]{0.64,0.00,0.00}{\textbf{#1}}}
\newcommand{\ExtensionTok}[1]{#1}
\newcommand{\FloatTok}[1]{\textcolor[rgb]{0.00,0.00,0.81}{#1}}
\newcommand{\FunctionTok}[1]{\textcolor[rgb]{0.00,0.00,0.00}{#1}}
\newcommand{\ImportTok}[1]{#1}
\newcommand{\InformationTok}[1]{\textcolor[rgb]{0.56,0.35,0.01}{\textbf{\textit{#1}}}}
\newcommand{\KeywordTok}[1]{\textcolor[rgb]{0.13,0.29,0.53}{\textbf{#1}}}
\newcommand{\NormalTok}[1]{#1}
\newcommand{\OperatorTok}[1]{\textcolor[rgb]{0.81,0.36,0.00}{\textbf{#1}}}
\newcommand{\OtherTok}[1]{\textcolor[rgb]{0.56,0.35,0.01}{#1}}
\newcommand{\PreprocessorTok}[1]{\textcolor[rgb]{0.56,0.35,0.01}{\textit{#1}}}
\newcommand{\RegionMarkerTok}[1]{#1}
\newcommand{\SpecialCharTok}[1]{\textcolor[rgb]{0.00,0.00,0.00}{#1}}
\newcommand{\SpecialStringTok}[1]{\textcolor[rgb]{0.31,0.60,0.02}{#1}}
\newcommand{\StringTok}[1]{\textcolor[rgb]{0.31,0.60,0.02}{#1}}
\newcommand{\VariableTok}[1]{\textcolor[rgb]{0.00,0.00,0.00}{#1}}
\newcommand{\VerbatimStringTok}[1]{\textcolor[rgb]{0.31,0.60,0.02}{#1}}
\newcommand{\WarningTok}[1]{\textcolor[rgb]{0.56,0.35,0.01}{\textbf{\textit{#1}}}}

%UL set white space before and after code blocks
\renewenvironment{Shaded}
{
  \vspace{10pt}%
  \begin{snugshade}%
}{%
  \end{snugshade}%
  \vspace{8pt}%
}

% User-included things with header_includes or in_header will appear here
% kableExtra packages will appear here if you use library(kableExtra)
\usepackage{booktabs}
\usepackage{longtable}
\usepackage{array}
\usepackage{multirow}
\usepackage{wrapfig}
\usepackage{float}
\usepackage{colortbl}
\usepackage{pdflscape}
\usepackage{tabu}
\usepackage{threeparttable}
\usepackage{threeparttablex}
\usepackage[normalem]{ulem}
\usepackage{makecell}
\usepackage{xcolor}


%UL set section header spacing
\usepackage{titlesec}
% 
\titlespacing\subsubsection{0pt}{24pt plus 4pt minus 2pt}{0pt plus 2pt minus 2pt}


%UL set whitespace around verbatim environments
\usepackage{etoolbox}
\makeatletter
\preto{\@verbatim}{\topsep=0pt \partopsep=0pt }
\makeatother



%%%%%%% PAGE HEADERS AND FOOTERS %%%%%%%%%
\usepackage{fancyhdr}
\setlength{\headheight}{15pt}
\fancyhf{} % clear the header and footers
\pagestyle{fancy}
\renewcommand{\chaptermark}[1]{\markboth{\thechapter. #1}{\thechapter. #1}}
\renewcommand{\sectionmark}[1]{\markright{\thesection. #1}} 
\renewcommand{\headrulewidth}{0pt}

\fancyhead[LO]{\emph{\leftmark}} 
\fancyhead[RE]{\emph{\rightmark}} 

% UL page number position 
\fancyfoot[C]{\emph{\thepage}} %regular pages
\fancypagestyle{plain}{\fancyhf{}\fancyfoot[C]{\emph{\thepage}}} %chapter pages

% JEM fix header on cleared pages for openright
\def\cleardoublepage{\clearpage\if@twoside \ifodd\c@page\else
   \hbox{}
   \fancyfoot[C]{}
   \newpage
   \if@twocolumn\hbox{}\newpage
   \fi
   \fancyhead[LO]{\emph{\leftmark}} 
   \fancyhead[RE]{\emph{\rightmark}} 
   \fi\fi}


%%%%% SELECT YOUR DRAFT OPTIONS
% This adds a "DRAFT" footer to every normal page.  (The first page of each chapter is not a "normal" page.)

% IP feb 2021: option to include line numbers in PDF

% This highlights (in blue) corrections marked with (for words) \mccorrect{blah} or (for whole
% paragraphs) \begin{mccorrection} . . . \end{mccorrection}.  This can be useful for sending a PDF of
% your corrected thesis to your examiners for review.  Turn it off, and the blue disappears.
\correctionstrue


%%%%% BIBLIOGRAPHY SETUP
% Note that your bibliography will require some tweaking depending on your department, preferred format, etc.
% If you've not used LaTeX before, I recommend reading a little about biblatex/biber and getting started with it.
% If you're already a LaTeX pro and are used to natbib or something, modify as necessary.
% Either way, you'll have to choose and configure an appropriate bibliography format...


\usepackage[style=authoryear, sorting=nyt, backend=biber, maxcitenames=2, useprefix, doi=true, isbn=false, uniquename=false]{biblatex}
\newcommand*{\bibtitle}{Works Cited}

\addbibresource{references.bib}


% This makes the bibliography left-aligned (not 'justified') and slightly smaller font.
\renewcommand*{\bibfont}{\raggedright\small}


% Uncomment this if you want equation numbers per section (2.3.12), instead of per chapter (2.18):
%\numberwithin{equation}{subsection}


%%%%% THESIS / TITLE PAGE INFORMATION
% Everybody needs to complete the following:
\title{\texttt{oxforddown}:\\
An Oxford University Thesis\\
Template for R Markdown}
\author{Author Name}
\college{Your College}

% Master's candidates who require the alternate title page (with candidate number and word count)
% must also un-comment and complete the following three lines:

% Uncomment the following line if your degree also includes exams (eg most masters):
%\renewcommand{\submittedtext}{Submitted in partial completion of the}
% Your full degree name.  (But remember that DPhils aren't "in" anything.  They're just DPhils.)
\degree{Doctor of Philosophy}
% Term and year of submission, or date if your board requires (eg most masters)
\degreedate{Michaelmas 2018}


%%%%% YOUR OWN PERSONAL MACROS
% This is a good place to dump your own LaTeX macros as they come up.

% To make text superscripts shortcuts
	\renewcommand{\th}{\textsuperscript{th}} % ex: I won 4\th place
	\newcommand{\nd}{\textsuperscript{nd}}
	\renewcommand{\st}{\textsuperscript{st}}
	\newcommand{\rd}{\textsuperscript{rd}}

%%%%% THE ACTUAL DOCUMENT STARTS HERE
\begin{document}

%%%%% CHOOSE YOUR LINE SPACING HERE
% This is the official option.  Use it for your submission copy and library copy:
\setlength{\textbaselineskip}{22pt plus2pt}
% This is closer spacing (about 1.5-spaced) that you might prefer for your personal copies:
%\setlength{\textbaselineskip}{18pt plus2pt minus1pt}

% You can set the spacing here for the roman-numbered pages (acknowledgements, table of contents, etc.)
\setlength{\frontmatterbaselineskip}{17pt plus1pt minus1pt}

% UL: You can set the line and paragraph spacing here for the separate abstract page to be handed in to Examination schools
\setlength{\abstractseparatelineskip}{13pt plus1pt minus1pt}
\setlength{\abstractseparateparskip}{0pt plus 1pt}

% UL: You can set the general paragraph spacing here - I've set it to 2pt (was 0) so
% it's less claustrophobic
\setlength{\parskip}{2pt plus 1pt}

%
% Oxford University logo on title page
%
\def\crest{{\includegraphics[width=5cm]{templates/beltcrest.pdf}}}
\renewcommand{\university}{University of Oxford}
\renewcommand{\submittedtext}{A thesis submitted for the degree of}


% Leave this line alone; it gets things started for the real document.
\setlength{\baselineskip}{\textbaselineskip}


%%%%% CHOOSE YOUR SECTION NUMBERING DEPTH HERE
% You have two choices.  First, how far down are sections numbered?  (Below that, they're named but
% don't get numbers.)  Second, what level of section appears in the table of contents?  These don't have
% to match: you can have numbered sections that don't show up in the ToC, or unnumbered sections that
% do.  Throughout, 0 = chapter; 1 = section; 2 = subsection; 3 = subsubsection, 4 = paragraph...

% The level that gets a number:
\setcounter{secnumdepth}{2}
% The level that shows up in the ToC:
\setcounter{tocdepth}{1}


%%%%% ABSTRACT SEPARATE
% This is used to create the separate, one-page abstract that you are required to hand into the Exam
% Schools.  You can comment it out to generate a PDF for printing or whatnot.

% JEM: Pages are roman numbered from here, though page numbers are invisible until ToC.  This is in
% keeping with most typesetting conventions.
\begin{romanpages}

% Title page is created here
\maketitle

%%%%% DEDICATION -- If you'd like one, un-comment the following.
\begin{dedication}
  For Yihui Xie
\end{dedication}

%%%%% ACKNOWLEDGEMENTS -- Nothing to do here except comment out if you don't want it.
\begin{acknowledgements}
 	This is where you will normally thank your advisor, colleagues, family and friends, as well as funding and institutional support. In our case, we will give our praises to the people who developed the ideas and tools that allow us to push open science a little step forward by writing plain-text, transparent, and reproducible theses in R Markdown.

  We must be grateful to John Gruber for inventing the original version of Markdown, to John MacFarlane for creating Pandoc (\url{http://pandoc.org}) which converts Markdown to a large number of output formats, and to Yihui Xie for creating \texttt{knitr} which introduced R Markdown as a way of embedding code in Markdown documents, and \texttt{bookdown} which added tools for technical and longer-form writing.

  Special thanks to \href{http://chester.rbind.io}{Chester Ismay}, who created the \texttt{thesisdown} package that helped many a PhD student write their theses in R Markdown. And a very special thanks to John McManigle, whose adaption of Sam Evans' adaptation of Keith Gillow's original maths template for writing an Oxford University DPhil thesis in LaTeX provided the template that I in turn adapted for R Markdown.

  Finally, profuse thanks to JJ Allaire, the founder and CEO of \href{http://rstudio.com}{RStudio}, and Hadley Wickham, the mastermind of the tidyverse without whom we'd all just given up and done data science in Python instead. Thanks for making data science easier, more accessible, and more fun for us all.

  \begin{flushright}
  Ulrik Lyngs \\
  Linacre College, Oxford \\
  2 December 2018
  \end{flushright}
\end{acknowledgements}


%%%%% ABSTRACT -- Nothing to do here except comment out if you don't want it.
\begin{abstract}
	This \emph{R Markdown} template is for writing an Oxford University thesis. The template is built using Yihui Xie's \texttt{bookdown} package, with heavy inspiration from Chester Ismay's \texttt{thesisdown} and the \texttt{OxThesis} \LaTeX~template (most recently adapted by John McManigle).

 This template's sample content include illustrations of how to write a thesis in R Markdown, and largely follows the structure from \href{https://ulyngs.github.io/rmarkdown-workshop-2019/}{this R Markdown workshop}.

 Congratulations for taking a step further into the lands of open, reproducible science by writing your thesis using a tool that allows you to transparently include tables and dynamically generated plots directly from the underlying data. Hip hooray!
\end{abstract}

%%%%% MINI TABLES
% This lays the groundwork for per-chapter, mini tables of contents.  Comment the following line
% (and remove \minitoc from the chapter files) if you don't want this.  Un-comment either of the
% next two lines if you want a per-chapter list of figures or tables.
  \dominitoc % include a mini table of contents

% This aligns the bottom of the text of each page.  It generally makes things look better.
\flushbottom

% This is where the whole-document ToC appears:
\tableofcontents

\listoffigures
	\mtcaddchapter
  	% \mtcaddchapter is needed when adding a non-chapter (but chapter-like) entity to avoid confusing minitoc

% Uncomment to generate a list of tables:
\listoftables
  \mtcaddchapter
%%%%% LIST OF ABBREVIATIONS
% This example includes a list of abbreviations.  Look at text/abbreviations.tex to see how that file is
% formatted.  The template can handle any kind of list though, so this might be a good place for a
% glossary, etc.
% First parameter can be changed eg to "Glossary" or something.
% Second parameter is the max length of bold terms.
\begin{mclistof}{List of Abbreviations}{3.2cm}

\item[1-D, 2-D]

One- or two-dimensional, referring \textbf{in this thesis} to spatial dimensions in an image.

\item[Otter]

One of the finest of water mammals.

\item[Hedgehog]

Quite a nice prickly friend.

\end{mclistof} 


% The Roman pages, like the Roman Empire, must come to its inevitable close.
\end{romanpages}

%%%%% CHAPTERS
% Add or remove any chapters you'd like here, by file name (excluding '.tex'):
\flushbottom

% all your chapters and appendices will appear here
\hypertarget{intro}{%
\chapter{Einleitung}\label{intro}}

Vergangene und gegenwärtige Bilanzskandale wie ENRON, Worldcom oder Wirecard führen immer wieder zu Diskussionen über die Vertrauenswürdigkeit des Kapitalmarkts sowie über die Vertrauenswürdigkeit der Abschlussprüfung. Bilanzskandale führen weit-reichende Konsequenten mit sich welche eine genauere Analyse des Warums und wie anstoßen (vgl. Boecker/Zwirner 2012, S. 1).
Diese Arbeit hat das Ziel, den Einsatz neuronaler Netze gegenüber der gängigeren Me-thode der logistischen Regression (vgl. Bao et al., 2020) zu vergleichen und dabei her-auszufinden, ob erstere signifikant besser darin sind, so viele Betrugsfälle wie möglich zu identifizieren, ohne die triviale Annahme zu treffen, dass jeder Fall ein Betrugsfall ist. Bei der Modellevaluation wird angenommen, dass ein nicht-identifizierter Betrugsfall doppelt so schwer wiegt, wie ein prognostizierter Betrugsfall, der sich als Nicht-Betrug herausstellt.
Die Metrik zur Messung der Ergebnisse ist der \(F_{\beta}\)-Score (vgl. Tharwat 2020, S. 174). Dieser bildet das harmonische Mittel aus Präzision und Sensitivität. Die Präzision sagt aus, wie viele vorhergesagte Betrugsfälle tatsächlich Betrugsfälle sind, wobei die Sensi-tivität dabei auf die Frage antwortet, wie viele der tatsächlichen Betrugsfälle als solche identifiziert wurden. Weil die Sensitivität im Kontext der Fraud Detection bedeutungs-voller erscheint, wird sie in dieser Arbeit doppelt so hoch gewichtet wie die Präzision. Die Formel dieser Metrik lautet dabei:
\[F_{\beta = 2} = (1+\beta^{2})*\frac{P*R}{\beta * P*R}\]

\hypertarget{AccFraud_and_ML}{%
\chapter{Accounting Fraud und Machine Learning}\label{AccFraud_and_ML}}

Das Handelsgesetzbuch (HGB) sieht gemäß § 317 Absatz (Abs.) 1 Satz (S.) 3 vor, dass der Abschlussprüfer im Rahmen der Abschlussprüfung, Unrichtigkeiten und Verstöße welche der Ordnungsmäßigkeit des Abschlusses entgegen stehen, erkennt und entsprechend deklariert. Was unter den Begrifflichkeiten Unrichtigkeit und Verstöße zu verstehen ist, wird durch den Gesetzgeber an dieser Stelle nicht weiter konkretisiert (vgl. Zwernemann et al.~2015, S. 22; § 317 Abs.1 HGB).

Dem entgegen versucht das Institut der Wirtschaftsprüfer (IDW) mit dem veröffentlichten Prüfungsstandard 210 (IDW PS 210) Rechnung zu tragen. Diesem zu entnehmen ist, dass ein Fehlerhafter Abschluss entweder auf Fraud (Verstoß) oder Error (Unrichtigkeit) zurückzuführen ist. Unter dem Begriff Unrichtigkeit wird eine unabsichtliche Angabe im Abschluss verstanden. Konkret bedeutet dies begangene Rechenfehler, eine unbewusst falsche Anwendung von Rechnungslegungsgrundsätzen sowie die falsche Einschätzung von Sachverhalten (vgl. Hlavica et al.~2016, S. 209f.). Der Begriff Verstoß dagegen umfasst eine beabsichtigte Handlung mit dem Ziel rechtswidrige Vorteile zu realisieren. Diese Handlungen konkretisiert der IDW PS 210 als Vermögensschädigungen, Täuschungen und Gesetzesverstöße, welche eine Auswirkung auf die Rechnungslegung zur Folge haben (vgl. Zwernemann et al.~2015, S. 8).
Um die Gründe für eine betrügerische Handlung nachvollziehen zu können, entwickelte Donald Cressey in den 1940er Jahren das sogenannte „Fraud-Triangle``. Dieses Dreieck wird ferner dem IDW PS 210 zugrunde gelegt (vgl. Boecker/Zwirner 2012, S. 2f.). Demnach tritt ein Verstoße dann auf, wenn drei Gegebenheiten als erfüllt angesehen werden können. So muss der Täter eine Gelegenheit zu der Tat haben und einen Anreiz (Motivation) für die Tat verspüren. Als letztes muss der Täter die Tat als moralisch akzeptabel rechtfertigen vor sich selbst rechtfertigen (vgl. Schuchter/Levi 2016, S. 3f.).
Aber nicht nur durch psychologische Ansätze versucht die Wissenschaft Verstöße einzuordnen und zu identifizieren, sondern auch durch eine Vielzahl an Machine Learning Ansätzen, welche Verstöße mittels Algorithmen identifizieren sollen.
Vor dem Hintergrund der potenziellen Gefahren des Bilanzbetrugs werden Letztere zunehmend für die Vorhersage und Aufdeckung von diskretionärer Bilanzpolitik eingesetzt. Die Benchmark in diesem Bereich ist das Dechow et al.~Modell, welches auf Grundlage von Accounting and Auditing Enforcement Releases (AAERs) der U.S. Securities and Exchange Comission (SEC) mit Hilfe einer logistischen Regression die Wahrscheinlichkeiten von (bewusst) fehlerhaften Darstellungen schätzt und klassifiziert (Dechow et al.~2011). Hierbei gelten die AAERs als ProxyVariable für die Manipulation der Bilanz. Durch das Voraussetzen der Untersuchungshandlungen seitens der SEC ergibt sich der Vorteil, dass der Typ I Fehler -- das Modell sagt fälschlicherweise ein misstatement voraus -- deutlich geringer ausfällt. Durch einige wenige Transformationen der logistischen Funktion kann der Einfluss einer jeden unabhängigen Variable durch den entsprechenden Regressionskoeffizienten hinsichtlich der Effektgröße verglichen werden, weswegen die Ergebnisse gut interpretierbar sind. Aus dem Dechow et al.~Modell folgt eine korrekte Klassifizierung von misstatements und nonmisstatements von ungefähr 63\% (Dechow et al.~2011, S.59). Die Sensitivität, d.h. in wie vielen Fällen das Modell einen misstatement richtig vorhergesagt hat, liegt bei etwas mehr als 68\% (339 von 494). Der Typ II Fehler (das Modell klassifiziert ein misstatement als nonmisstatement), der im Rahmen des accounting frauds schwerwiegender ist als der Typ I Fehler (vgl. Lin et al.~2015, S. 468), liegt bei etwas mehr als 31\% (155 von 494).
Ein Vergleich der Performance der logistischen Regression mit den neuronalen Netzen, einer weiteren Methode zur Vorhersage von Bilanzbetrug, findet sich in dem Paper von Lin et al.~(2015). Aus diesem geht hervor, dass die neuronalen Netze hinsichtlich der Aufdeckung von accounting fraud bessere Ergebnisse liefern als die logistische Regression. Die artificial neural networks (ANNs) erreichen bei dem Testdatensatz eine Genauigkeit von fast 93\%. Die Sensitivität von fast 83\% liegt zudem deutlich höher als bei der logistischen Regression, bei der 72\% der misstatements richtig vorhergesagt wurden (vgl. Lin et al.~2015, S. 465f.).
Die Interpretierbarkeit und verhältnismäßig einfache Anwendbarkeit haben die logistische Regression zu einem beliebten Instrument gemacht, die Ergebnisse hinsichtlich der Vorhersage von Bilanzbetrug werden allerdings von anderen Modellen übertroffen (vgl. Dutta et al.~2017, S. 375). Im Falle der neuronalen Netze ergibt sich wiederum der Nachteil der geringeren Transparenz hinsichtlich der Arbeitsweise des Algorithmus (vgl. Bao et al.~2020, S. 228). Schlussendlich ergibt sich ein Trade-Off zwischen der Interpretierbarkeit und Vorhersagekraft. Nachfolgend liegt der Fokus dieser Ausarbeitung auf der reinen Performance respektive Vorhersagekraft des Modells. Ein Modell, welches misstatements richtig vorhersagt, erscheint bei der Klassifizierung von Bilanzbetrug wichtiger als ein Modell, aus dem abgelesen werden kann, welche Variablen den misstatement wie stark beeinflussen. An dieser Stelle sollte erwähnt werden, dass auch ein Modell, welches in 99\% der Fälle die richtige Vorhersage trifft, hinsichtlich der Aufdeckung von accounting fraud nicht geeignet sein muss. Durch die Problematik der signifikanten sample imbalance -- Betrugsfälle sind stark unterrepräsentiert -- ist es möglich, dass misstatements durch das Modell nicht erkannt respektive falsch klassifiziert werden, also Typ II Fehler auftreten können.

\hypertarget{dataset}{%
\chapter{Datensatz}\label{dataset}}

Ein besonderer Fokus liegt aus diesem Grund auf der Minimierung des Typ II Fehlers respektive der Maximierung der Sensitivität des Modells. Eine höchstmögliche Genauigkeit ist gut, aufgrund der sehr geringen Anzahl an bilanziellen Verfehlungen verglichen zu der Stichprobengröße sollte hier allerdings kein Schwerpunkt liegen.
Zum Training und Test des Machine Learning Modells, wird ein Datensatz aus der Veröffentlichung von Bao et al.~(2020) verwendet. Die Autoren haben diesen im Internet auf der Seite „GitHub`` zur Verfügung gestellt (vgl. Bao et al.~2020, GitHub Repository). Dieser besteht aus allen öffentlich gelisteten US-amerikanischen Firmen im Zeitraum von 1991 bis 2008. Die Accounting-Betrugsfälle aus den „Accounting and Auditing Enforcement Releases`` (AAER), die von der United States Securities and Exchange Commission (SEC) im gleichen Zeitraum veröffentlicht worden sind (vgl. Bao et al.~2020, S. 207). Der Datensatz listet für jeden Eintrag 28 verschiedene finanzielle Items auf. Diese setzen sich aus den Veröffentlichungen von Cecchini et al.~(2010) und Dechow et al.~(2011) zusammen. Die finanziellen Items stammen aus vier verschiedenen Bereichen, der Bilanz (z.B. gesamte Forderungen), der Gewinn- und Verlustrechnung (z.B. Nettoumsatz), der Kapitalflussrechnung (z.B. Langzeitemission von Schuldtiteln) und dem Marktwert (z.B. Common Shares Outstanding).
Da Accounting-Betrugsfälle eher seltener vorkommen (vgl. Dutta et al.~2011, S. 381), weist der Datensatz eine große Verteilungsungleichheit zwischen den Betrugs- und Nicht-betrugsfällen auf. Weniger als ein Prozent aller Einträge im Datensatz sind hierbei Betrugsfälle. Weitere Betrugsfälle könnten über andere Datenbanken gesucht werden, hierbei gibt es allerdings das Problem, dass die Betrugsfälle identifiziert und einzeln herausgesucht werden müssen. Daher besteht hier ein „class imbalance`` Problem, das mithilfe des Machine Learning Algorithmus gelöst werden muss.
Zudem sind nur Betrugsfälle bis zum Jahr 2008 in diesem Datensatz erhalten, da die SEC nach der Finanzkrise ihre Prioritäten geändert hat (vgl. Bao et al.~2020, S. 208). Sollte sich die Art und Weise mit der Accounting-Betrug durchgeführt wird in den Jahren danach geändert haben, so können diese Fälle unter Umständen nicht vom Algorithmus erkannt werden.

\hypertarget{methods}{%
\chapter{Methoden}\label{methods}}

Diese Arbeit vergleicht die Vorhersagequalität der in der Accounting Fraud Detection gängigen logistischen Regression nach Dechow et al.~(vgl. Bao et al.~2020, S. 2) mit der von neuronalen Netzen.
Die logistische Regression ist eng mit der linearen Regression verwandt und wird zur binären Schätzung einer Klassenzugehörigkeit verwandt (vgl. Géron 2019, S. 144). Dabei berechnet sie eine gewichtete Summe von Inputfaktoren und aggregiert sie zu einer Wahrscheinlichkeit zwischen 0 und 1. Liegt diese Wahrscheinlichkeit bei mindestens 0.5, so wird die Klasse „1`` vorhergesagt, welcher in dieser Arbeit der Klasse „fraud`` entspricht (vgl. Géron 2019, S. 144). Hierzu wird für jedes Attribut einer Beobachtung eine Sigmoid-Funktion verwendet, welche S-förmig vom Minimum bis zum Maximum des jeweiligen Attributs verläuft und die Verteilungen der Merkmalsausprägungen möglichst gut nach Klassenzugehörigkeit abgrenzt. Je weiter eine Merkmalsausprägung von dieser Grenze entfernt ist, desto näher ist der Funktionswert an der 1 oder der 0 (vgl. Géron 2019, S. 148).
Neuronale Netze bestehen aus drei Sorten von Schichten: Input-Schichten, welche Daten einlesen, versteckte Schichten, welche die Daten verarbeiten und Output-Schichten, welche aus den verarbeiteten Daten eine Prognose ableiten (vgl. Géron 2019, S. 286). In dieser Arbeit besteht die Output-Schicht aus lediglich einem Knoten, welche die Klassen „fraud`` abbildet. Die Anzahl der Knoten der Input-Schicht entspricht der Anzahl der Variablen im Datensatz. Die Knoten einer Schicht sind jeweils mit jedem Knoten seiner nachfolgenden Schicht durch „Gewichte`` verbunden. Jeder einzelne Knoten aggregiert die Signale, die er empfängt über deren Gewichte zu einer Zahl und wendet eine Aktivierungsfunktion an (vgl. Géron 2019, S. 282). Übersteigt der Funktionswert einen gegeben Schwellenwert, so „feuert`` das Neuron, was bedeutet, dass es ein Signal größer 0 an die Neuronen der nächsten Schicht weitergibt (vgl. Géron 2019, S. 282-283). Die Gewichte und alle Schwellenwerte werden durch Backpropagation unter Zuhilfenahme des Gradient Descent Algorithmus verbessert (vgl. Géron 2019, S.119 und S, 286). Sind alle Trainingsdaten einmal zum Training herangezogen worden, bedeutet das, dass das Netz für „eine Epoche`` trainiert wurde (vgl. Géron 2019, S. 127). In dieser Arbeit wird ein Netz über 100 Epochen hinweg trainiert.

\begin{Shaded}
\begin{Highlighting}[]
\FunctionTok{library}\NormalTok{(readr)}
\NormalTok{data }\OtherTok{\textless{}{-}} \FunctionTok{read\_csv}\NormalTok{(}\StringTok{"data/data.csv"}\NormalTok{)}
\end{Highlighting}
\end{Shaded}

\begin{verbatim}
## 
## -- Column specification --------------------------------------------------------
## cols(
##   .default = col_double()
## )
## i Use `spec()` for the full column specifications.
\end{verbatim}

\begin{Shaded}
\begin{Highlighting}[]
\CommentTok{\# str(data)}

\FunctionTok{dim}\NormalTok{(data)}
\end{Highlighting}
\end{Shaded}

\begin{verbatim}
## [1] 31990    51
\end{verbatim}

Wir haben 31990 Beobachtungen auf 51 Variablen. Wir müssen die fehlenden Werte Ientifizieren und die Spalten, welche zu viele davon enthalten löschen. Danach löschen wir die übrigen Zeilen, welche NaNs enthalten. Dies dient dem Umstand, dass NN mit NaN-Werten nicht arbeiten können. Wir verwenden für LogReg und NN den glichen Datensatz, um die Performance der Modelle vergleichen zu können.

\begin{Shaded}
\begin{Highlighting}[]
\ControlFlowTok{for}\NormalTok{ (col }\ControlFlowTok{in} \FunctionTok{colnames}\NormalTok{(data))\{}
\NormalTok{  nan\_number }\OtherTok{=} \FunctionTok{sum}\NormalTok{(}\FunctionTok{is.na}\NormalTok{(data[col]))}
  \FunctionTok{print}\NormalTok{(}\FunctionTok{paste}\NormalTok{(col, nan\_number ))}
\NormalTok{\}}
\end{Highlighting}
\end{Shaded}

\begin{verbatim}
## [1] "fyear 0"
## [1] "gvkey 0"
## [1] "sich 199"
## [1] "insbnk 0"
## [1] "understatement 0"
## [1] "option 0"
## [1] "p_aaer 31846"
## [1] "new_p_aaer 31846"
## [1] "misstate 0"
## [1] "act 0"
## [1] "ap 0"
## [1] "at 0"
## [1] "ceq 0"
## [1] "che 0"
## [1] "cogs 0"
## [1] "csho 0"
## [1] "dlc 0"
## [1] "dltis 0"
## [1] "dltt 0"
## [1] "dp 0"
## [1] "ib 0"
## [1] "invt 0"
## [1] "ivao 0"
## [1] "ivst 0"
## [1] "lct 0"
## [1] "lt 0"
## [1] "ni 0"
## [1] "ppegt 0"
## [1] "pstk 0"
## [1] "re 0"
## [1] "rect 0"
## [1] "sale 0"
## [1] "sstk 0"
## [1] "txp 0"
## [1] "txt 0"
## [1] "xint 0"
## [1] "prcc_f 0"
## [1] "dch_wc 723"
## [1] "ch_rsst 734"
## [1] "dch_rec 678"
## [1] "dch_inv 681"
## [1] "soft_assets 13"
## [1] "ch_cs 3631"
## [1] "ch_cm 3763"
## [1] "ch_roa 3380"
## [1] "issue 0"
## [1] "bm 6"
## [1] "dpi 1084"
## [1] "reoa 13"
## [1] "EBIT 13"
## [1] "ch_fcf 836"
\end{verbatim}

Die Variable, ``p\_aaer'' und ``new\_p\_aaer'' fallen durch extrem viele NaN-Werte auf und werden komplett entfernt. Die übrigen Variablen bestehen zu maximal ca 10\% aus NaN-Werten und werden daher als unbedenklich eingestuft. ``new\_p\_aaer'' und alle Beobachtungen, welche NaN-Werte enthalten, werden beseitigt.

\begin{Shaded}
\begin{Highlighting}[]
\NormalTok{data }\OtherTok{\textless{}{-}}\NormalTok{ data[, }\SpecialCharTok{{-}}\FunctionTok{match}\NormalTok{(}\FunctionTok{c}\NormalTok{(}\StringTok{"new\_p\_aaer"}\NormalTok{, }\StringTok{"p\_aaer"}\NormalTok{), }\FunctionTok{names}\NormalTok{(data))]}
\NormalTok{data }\OtherTok{\textless{}{-}}\NormalTok{ data[}\FunctionTok{complete.cases}\NormalTok{(data),]}
\FunctionTok{dim}\NormalTok{(data)}
\end{Highlighting}
\end{Shaded}

\begin{verbatim}
## [1] 27792    49
\end{verbatim}

\begin{Shaded}
\begin{Highlighting}[]
\FunctionTok{print}\NormalTok{(data[}\DecValTok{5}\NormalTok{,])}
\end{Highlighting}
\end{Shaded}

\begin{verbatim}
## # A tibble: 1 x 49
##   fyear gvkey  sich insbnk understatement option misstate   act    ap    at
##   <dbl> <dbl> <dbl>  <dbl>          <dbl>  <dbl>    <dbl> <dbl> <dbl> <dbl>
## 1  1990  1028  7385      0              0      0        0  17.3  3.52  27.5
## # ... with 39 more variables: ceq <dbl>, che <dbl>, cogs <dbl>, csho <dbl>,
## #   dlc <dbl>, dltis <dbl>, dltt <dbl>, dp <dbl>, ib <dbl>, invt <dbl>,
## #   ivao <dbl>, ivst <dbl>, lct <dbl>, lt <dbl>, ni <dbl>, ppegt <dbl>,
## #   pstk <dbl>, re <dbl>, rect <dbl>, sale <dbl>, sstk <dbl>, txp <dbl>,
## #   txt <dbl>, xint <dbl>, prcc_f <dbl>, dch_wc <dbl>, ch_rsst <dbl>,
## #   dch_rec <dbl>, dch_inv <dbl>, soft_assets <dbl>, ch_cs <dbl>, ch_cm <dbl>,
## #   ch_roa <dbl>, issue <dbl>, bm <dbl>, dpi <dbl>, reoa <dbl>, EBIT <dbl>,
## #   ch_fcf <dbl>
\end{verbatim}

Damit ein NN Inputdaten verwerten kann, müssen diese zwischen 0 und 1 normiert sein.

\begin{Shaded}
\begin{Highlighting}[]
\NormalTok{normalize }\OtherTok{\textless{}{-}} \ControlFlowTok{function}\NormalTok{(x, }\AttributeTok{na.rm =} \ConstantTok{TRUE}\NormalTok{) \{}
    \FunctionTok{return}\NormalTok{((x}\SpecialCharTok{{-}} \FunctionTok{min}\NormalTok{(x)) }\SpecialCharTok{/}\NormalTok{(}\FunctionTok{max}\NormalTok{(x)}\SpecialCharTok{{-}}\FunctionTok{min}\NormalTok{(x)))}
\NormalTok{\}}
\CommentTok{\# as.data.frame(apply(df$name, normalize))}
\ControlFlowTok{for}\NormalTok{ (col }\ControlFlowTok{in} \FunctionTok{colnames}\NormalTok{(data))}
\NormalTok{\{}
\NormalTok{  data[col] }\OtherTok{\textless{}{-}} \FunctionTok{normalize}\NormalTok{(data[col])}
\NormalTok{\}}


\FunctionTok{print}\NormalTok{(data[}\DecValTok{5}\NormalTok{,])}
\end{Highlighting}
\end{Shaded}

\begin{verbatim}
## # A tibble: 1 x 49
##   fyear    gvkey  sich insbnk understatement option misstate      act       ap
##   <dbl>    <dbl> <dbl>  <dbl>          <dbl>  <dbl>    <dbl>    <dbl>    <dbl>
## 1     0 0.000156 0.736      0              0      0        0 0.000262 0.000180
## # ... with 40 more variables: at <dbl>, ceq <dbl>, che <dbl>, cogs <dbl>,
## #   csho <dbl>, dlc <dbl>, dltis <dbl>, dltt <dbl>, dp <dbl>, ib <dbl>,
## #   invt <dbl>, ivao <dbl>, ivst <dbl>, lct <dbl>, lt <dbl>, ni <dbl>,
## #   ppegt <dbl>, pstk <dbl>, re <dbl>, rect <dbl>, sale <dbl>, sstk <dbl>,
## #   txp <dbl>, txt <dbl>, xint <dbl>, prcc_f <dbl>, dch_wc <dbl>,
## #   ch_rsst <dbl>, dch_rec <dbl>, dch_inv <dbl>, soft_assets <dbl>,
## #   ch_cs <dbl>, ch_cm <dbl>, ch_roa <dbl>, issue <dbl>, bm <dbl>, dpi <dbl>,
## #   reoa <dbl>, EBIT <dbl>, ch_fcf <dbl>
\end{verbatim}

Jetzt wird ein NN trainiert und eine erste Prognose abgegeben. Seed = 42

\begin{Shaded}
\begin{Highlighting}[]
\FunctionTok{library}\NormalTok{(nnet)}
\FunctionTok{set.seed}\NormalTok{(}\DecValTok{42}\NormalTok{)}
\NormalTok{inTrain }\OtherTok{\textless{}{-}} \FunctionTok{runif}\NormalTok{(}\FunctionTok{nrow}\NormalTok{(data)) }\SpecialCharTok{\textless{}} \FloatTok{0.2}

\NormalTok{model }\OtherTok{\textless{}{-}} \FunctionTok{nnet}\NormalTok{(misstate }\SpecialCharTok{\textasciitilde{}}\NormalTok{ ., }\AttributeTok{data=}\NormalTok{data[inTrain,],}
             \AttributeTok{size=}\DecValTok{15}\NormalTok{, }\AttributeTok{maxit=}\DecValTok{100}\NormalTok{, }\AttributeTok{rang=}\FloatTok{0.1}\NormalTok{, }\AttributeTok{decay=}\FloatTok{5e{-}4}\NormalTok{)}
\end{Highlighting}
\end{Shaded}

\begin{verbatim}
## # weights:  751
## initial  value 1761.201764 
## iter  10 value 33.571159
## iter  20 value 24.129969
## iter  30 value 24.030039
## iter  40 value 24.029168
## iter  50 value 24.028222
## iter  60 value 24.027147
## iter  70 value 24.025834
## iter  80 value 24.024014
## iter  90 value 24.020810
## iter 100 value 24.011132
## final  value 24.011132 
## stopped after 100 iterations
\end{verbatim}

\begin{Shaded}
\begin{Highlighting}[]
\NormalTok{pred }\OtherTok{\textless{}{-}} \FunctionTok{predict}\NormalTok{(model, data[}\SpecialCharTok{{-}}\NormalTok{inTrain,], }\AttributeTok{type=}\StringTok{"raw"}\NormalTok{) }
\end{Highlighting}
\end{Shaded}

Nun werden die Ergebnisse für einen Score verwendet.

\begin{Shaded}
\begin{Highlighting}[]
\NormalTok{cm }\OtherTok{\textless{}{-}} \FunctionTok{table}\NormalTok{(}\AttributeTok{pred=}\NormalTok{pred,}
\AttributeTok{true=}\NormalTok{data}\SpecialCharTok{$}\NormalTok{misstate[}\SpecialCharTok{{-}}\NormalTok{inTrain])}
\NormalTok{accuracy }\OtherTok{=} \FunctionTok{sum}\NormalTok{(}\FunctionTok{diag}\NormalTok{(cm))}\SpecialCharTok{/}\FunctionTok{sum}\NormalTok{(}\FunctionTok{sum}\NormalTok{(cm))}
\FunctionTok{print}\NormalTok{(accuracy)}
\end{Highlighting}
\end{Shaded}

\begin{verbatim}
## [1] 3.598287e-05
\end{verbatim}

Haha :D Verdammt, da lief irgendwas falsch. Ich denke, dass type=``raw'' bei der prediction quatsch ist und dass da stattdessen ``class'' rein müsste. Allerdings. funktioniert das dann aus Gründen nicht. morgen oder die Tage mal weiterprobieren.

\begin{savequote}
There is grandeur in this view of life, with its several powers, having
been originally breathed into a few forms or into one; and that, whilst
this planet has gone cycling on according to the fixed law of gravity,
from so simple a beginning endless forms most beautiful and most
wonderful have been, and are being, evolved.
\qauthor{--- Charles Darwin \autocite{Darwin1859}}\end{savequote}



\hypertarget{customisations-and-extensions}{%
\chapter{Customisations and extensions}\label{customisations-and-extensions}}

\minitoc 

\noindent This chapter describes a number of additional tips and tricks as well as possible customizations to the \texttt{oxforddown} thesis.

\hypertarget{front-matter}{%
\section{Front matter}\label{front-matter}}

\hypertarget{shorten-captions-shown-in-the-list-of-figures-pdf}{%
\subsection{Shorten captions shown in the list of figures (PDF)}\label{shorten-captions-shown-in-the-list-of-figures-pdf}}

You might want your list of figures (which follows the table of contents) to have shorter (or just different) figure descriptions than the actual figure captions.

Do this using the chunk option \texttt{fig.scap} (`short caption'), for example \texttt{\{r\ captain-image,\ fig.cap="A\ very\ long\ and\ descriptive\ (and\ potentially\ boring)\ caption\ that\ doesn\textquotesingle{}t\ fit\ in\ the\ list\ of\ figures,\ but\ helps\ the\ reader\ understand\ what\ the\ figure\ communicates.",\ fig.scap="A\ concise\ description\ for\ the\ list\ of\ figures"}

\hypertarget{shorten-captions-shown-in-the-list-of-tables-pdf}{%
\subsection{Shorten captions shown in the list of tables (PDF)}\label{shorten-captions-shown-in-the-list-of-tables-pdf}}

You might want your list of tables (which follows the list of figures in your thesis front matter) to have shorter (or just different) table descriptions than the actual table captions.

If you are using \texttt{knitr::kable} to generate a table, you can do this with the argument \texttt{caption.short}, e.g.:

\begin{Shaded}
\begin{Highlighting}[]
\NormalTok{knitr}\SpecialCharTok{::}\FunctionTok{kable}\NormalTok{(mtcars,}
              \AttributeTok{caption =} \StringTok{"A very long and descriptive (and potentially}
\StringTok{              boring) caption that doesn\textquotesingle{}t fit in the list of figures,}
\StringTok{              but helps the reader understand what the figure }
\StringTok{              communicates."}\NormalTok{,}
              \AttributeTok{caption.short =} \StringTok{"A concise description for the list of tables"}\NormalTok{)}
\end{Highlighting}
\end{Shaded}

\hypertarget{shorten-running-header-pdf}{%
\section{Shorten running header (PDF)}\label{shorten-running-header-pdf}}

You might want a chapter's running header (i.e.~the header showing the title of the current chapter at the top of page) to be shorter (or just different) to the actual chapter title.

Do this by adding the latex command \texttt{\textbackslash{}chaptermark\{My\ shorter\ version\}} after your chapter title.

For example, chapter \ref{cites-and-refs}`s running header is simply 'Cites and cross-refs', because it begins like this:

\begin{Shaded}
\begin{Highlighting}[]
\FunctionTok{\# Citations, cross{-}references, and collaboration \{\#cites{-}and{-}refs\} }
\NormalTok{\textbackslash{}chaptermark\{Cites and cross{-}refs\}}
\end{Highlighting}
\end{Shaded}

\hypertarget{unnumbered-chapters}{%
\section{Unnumbered chapters}\label{unnumbered-chapters}}

To make chapters unnumbered (normally only relevant to the Introduction and/or the Conclusion), follow the chapter header with \texttt{\{-\}}, e.g.~\texttt{\#\ Introduction\ \{-\}}.

When you do this, you must also follow the heading with these two latex commands:

\begin{Shaded}
\begin{Highlighting}[]
\FunctionTok{\textbackslash{}adjustmtc}
\FunctionTok{\textbackslash{}markboth}\NormalTok{\{The Name of Your Unnumbered Chapter\}\{\}}
\end{Highlighting}
\end{Shaded}

Otherwise the chapter's mini table of contents and the running header will show the previous chapter.

\hypertarget{beginning-chapters-with-quotes-pdf}{%
\section{Beginning chapters with quotes (PDF)}\label{beginning-chapters-with-quotes-pdf}}

The OxThesis LaTeX template lets you inject some wittiness into your thesis by including a block of type \texttt{savequote} at the beginning of chapters.
To do this, use the syntax \texttt{\textasciigrave{}\textasciigrave{}\textasciigrave{}\{block\ type=\textquotesingle{}savequote\textquotesingle{}\}}.\footnote{For more on custom block types, see the relevant section in \href{https://bookdown.org/yihui/bookdown/custom-blocks.html}{\emph{Authoring Books with R Markdown}}.}

Add the reference for the quote with the chunk option \texttt{quote\_author="my\ author\ name"}.
You will also want to add the chunk option \texttt{include=knitr::is\_latex\_output()} so that quotes are only included in PDF output.

It's not possible to use markdown syntax inside chunk options, so if you want to e.g.~italicise a book name in the reference use a \href{https://bookdown.org/yihui/bookdown/markdown-extensions-by-bookdown.html\#text-references}{`text reference'}: Create a named piece of text with `(ref:label-name) My text', then point to this in the chunk option with \texttt{quote\_author=\textquotesingle{}(ref:label-name)\textquotesingle{}}.

\hypertarget{highlighting-corrections-html-pdf}{%
\section{Highlighting corrections (HTML \& PDF)}\label{highlighting-corrections-html-pdf}}

For when it comes time to do corrections, you may want to highlight changes made when you submit a post-viva, corrected copy to your examiners so they can quickly verify you've completed the task.
You can do so like this:

\hypertarget{short-inline-corrections}{%
\subsection{Short, inline corrections}\label{short-inline-corrections}}

Highlight \textbf{short, inline corrections} by doing \texttt{{[}like\ this{]}\{.correction\}} --- the text between the square brackets will then \hl{be highlighted in blue} in the output.

Note that pandoc might get confused by citations and cross-references inside inline corrections.
In particular, it might get confused by \texttt{"{[}what\ @Shea2014\ said{]}\{.correction\}"} which becomes \autocite[what][ said]{Shea2014}\{.correction\}
In such cases, you can use LaTeX syntax directly.
The correction highlighting uses the \href{https://ctan.org/pkg/soul}{soul} package, so you can do like this:

\begin{itemize}
\tightlist
\item
  If using biblatex for references, use \texttt{"\textbackslash{}hl\{what\ \textbackslash{}textcite\{Shea2014\}\ said\}}
\item
  If using natbib for references, use \texttt{"\textbackslash{}hl\{what\ \textbackslash{}cite\{Shea2014\}\ said\}}
\end{itemize}

Using raw LaTeX has the drawback of corrections then not showing up in HTML output at all, but you might only care about correction highlighting in the PDF for your examiners anyway!

\hypertarget{blocks-of-added-or-changed-material}{%
\subsection{Blocks of added or changed material}\label{blocks-of-added-or-changed-material}}

Highlight entire \textbf{blocks of added or changed material} by putting them in a block of type \texttt{correction}, using the syntax \texttt{\textasciigrave{}\textasciigrave{}\textasciigrave{}\{block\ type=\textquotesingle{}correction\textquotesingle{}\}}.\footnote{In the \textbf{.tex} file for PDF output, this will put the content between \texttt{\textbackslash{}begin\{correction\}} and \texttt{\textbackslash{}end\{correction\}}; in gitbook output it will be put between \texttt{\textless{}div\ class="correction"\textgreater{}} and \texttt{\textless{}/div\textgreater{}}.}
Like so:

\begin{correction}
For larger chunks, like this paragraph or indeed entire figures, you can
use the \texttt{correction} block type. This environment
\textbf{highlights paragraph-sized and larger blocks} with the same blue
colour.
\end{correction}

\emph{Note that correction blocks cannot be included in word output.}

\hypertarget{stopping-corrections-from-being-highlighted}{%
\subsection{Stopping corrections from being highlighted}\label{stopping-corrections-from-being-highlighted}}

To turn off correction highlighting, go to the YAML header of \textbf{index.Rmd}, then:

\begin{itemize}
\tightlist
\item
  PDF output: set \texttt{corrections:\ false}\\
\item
  HTML output: remove or comment out \texttt{-\ templates/corrections.css}
\end{itemize}

\hypertarget{apply-custom-font-color-and-highlighting-to-text-html-pdf}{%
\section{Apply custom font color and highlighting to text (HTML \& PDF)}\label{apply-custom-font-color-and-highlighting-to-text-html-pdf}}

The lua filter that adds the functionality to highlight corrections adds two more tricks:
you can apply your own choice of colour to highlight text, or change the font color.
The syntax is as follows:

\begin{quote}
Here's \texttt{{[}some\ text\ in\ pink\ highlighting{]}\{highlight="pink"\}}\\
Becomes: Here's \sethlcolor{pink}\hl{some text in pink highlighting}\sethlcolor{correctioncolor}.
\end{quote}

\begin{quote}
\texttt{{[}Here\textquotesingle{}s\ some\ text\ with\ blue\ font{]}\{color="blue"\}}~\\
Becomes: \textcolor{blue}{Here's some text with blue font}
\end{quote}

\begin{quote}
Finally --- never, ever actually do this -- \texttt{{[}here\textquotesingle{}s\ some\ text\ with\ black\ highlighting\ and\ yellow\ font{]}\{highlight="black"\ color="yellow"\}}\\
Becomes: \textcolor{yellow}{\sethlcolor{black}\hl{here's some text with black highlighting and yellow font}\sethlcolor{correctioncolor}}
\end{quote}

The file \textbf{scripts\_and\_filters/colour\_and\_highlight.lua} implements this, if you want to fiddle around with it.
It works with both PDF and HTML output.

\hypertarget{embed-pdf}{%
\section{Including another paper in your thesis - embed a PDF document}\label{embed-pdf}}

You may want to embed existing PDF documents into the thesis, for example if your department allows a `portfolio' style thesis and you need to include an existing typeset publication as a chapter.

In gitbook output, you can simply use \texttt{knitr::include\_graphics} and it should include a scrollable (and downloadable) PDF.
You will probably want to set the chunk options \texttt{out.width=\textquotesingle{}100\%\textquotesingle{}} and \texttt{out.height=\textquotesingle{}1000px\textquotesingle{}}:

\begin{Shaded}
\begin{Highlighting}[]
\NormalTok{knitr}\SpecialCharTok{::}\FunctionTok{include\_graphics}\NormalTok{(}\StringTok{"figures/sample{-}content/pdf\_embed\_example/Lyngs2020\_FB.pdf"}\NormalTok{)}
\end{Highlighting}
\end{Shaded}

In LaTeX output, however, this approach can cause odd behaviour.
Therefore, when you build your thesis to PDF, split the PDF into an alphanumerically sorted sequence of \textbf{single-page} PDF files (you can do this automatically with the package \texttt{pdftools}). You can then use the appropriate LaTeX command to insert them, as shown below (for brevity, in the \texttt{oxforddown} PDF sample content we're only including two pages).
\emph{Note that the chunk option \texttt{results=\textquotesingle{}asis\textquotesingle{}} must be set.}
You may also want to remove margins from the PDF files, which you can do with Adobe Acrobat (paid version) and likely other software.

\begin{Shaded}
\begin{Highlighting}[]
\CommentTok{\# install.packages(pdftools)}
\CommentTok{\# split PDF into pages stored in}
\NormalTok{    figures}\SpecialCharTok{/}\NormalTok{sample}\SpecialCharTok{{-}}\NormalTok{content}\SpecialCharTok{/}\NormalTok{pdf\_embed\_example}\SpecialCharTok{/}\NormalTok{split}\SpecialCharTok{/}
\CommentTok{\#}
\NormalTok{    pdftools}\SpecialCharTok{::}\FunctionTok{pdf\_split}\NormalTok{(}\StringTok{"figures/sample{-}content/pdf\_embed\_example/Lyngs2020\_FB.pdf"}\NormalTok{,}
\CommentTok{\# output = "figures/sample{-}content/pdf\_embed\_example/split/")}

\CommentTok{\# grab the pages}
\NormalTok{pages }\OtherTok{\textless{}{-}} \FunctionTok{list.files}\NormalTok{(}\StringTok{"figures/sample{-}content/pdf\_embed\_example/split"}\NormalTok{,}
    \AttributeTok{full.names =} \ConstantTok{TRUE}\NormalTok{)}

\CommentTok{\# set how wide you want the inserted PDFs to be:}
\CommentTok{\# 1.0 is 100 per cent of the oxforddown PDF page width;}
\CommentTok{\# you may want to make it a bit bigger}
\NormalTok{pdf\_width }\OtherTok{\textless{}{-}} \FloatTok{1.2}

\CommentTok{\# for each PDF page, insert it nicely and}
\CommentTok{\# end with a page break}
\FunctionTok{cat}\NormalTok{(stringr}\SpecialCharTok{::}\FunctionTok{str\_c}\NormalTok{(}\StringTok{"}\SpecialCharTok{\textbackslash{}\textbackslash{}}\StringTok{newpage }\SpecialCharTok{\textbackslash{}\textbackslash{}}\StringTok{begin\{center\}}
\StringTok{    }\SpecialCharTok{\textbackslash{}\textbackslash{}}\StringTok{makebox[}\SpecialCharTok{\textbackslash{}\textbackslash{}}\StringTok{linewidth][c]\{}\SpecialCharTok{\textbackslash{}\textbackslash{}}\StringTok{includegraphics[width="}\NormalTok{, pdf\_width,}
    \StringTok{"}\SpecialCharTok{\textbackslash{}\textbackslash{}}\StringTok{linewidth]\{"}\NormalTok{, pages, }\StringTok{"\}\} }\SpecialCharTok{\textbackslash{}\textbackslash{}}\StringTok{end\{center\}"}\NormalTok{))}
\end{Highlighting}
\end{Shaded}

\newpage \begin{center} \makebox[\linewidth][c]{\includegraphics[width=1.2\linewidth]{figures/sample-content/pdf_embed_example/split/_000000000000001.pdf}} \end{center} \newpage \begin{center} \makebox[\linewidth][c]{\includegraphics[width=1.2\linewidth]{figures/sample-content/pdf_embed_example/split/_000000000000011.pdf}} \end{center}

\hypertarget{embed-rmd}{%
\section{Including another paper in your thesis - R Markdown child document}\label{embed-rmd}}

Sometimes you want to include another paper you are currently writing as a chapter in your thesis.
Above \ref{embed-pdf}, we described the simplest way to do this: include the other paper as a pdf.
However, in some cases you instead want to include the R Markdown source from this paper, and have it compiled within your thesis.
This is a little bit more tricky, because you need to keep careful track of your file paths, but it is possible by \href{https://bookdown.org/yihui/rmarkdown-cookbook/child-document.html}{including the paper as a child document}.
There are four main steps:

\begin{enumerate}
\def\labelenumi{\arabic{enumi}.}
\tightlist
\item
  Include the paper as a child document
\item
  Make file paths compatible with knitting the article on its own, as well as when it's include in your thesis
\item
  Make header levels correct
\item
  Make figure widths correct
\end{enumerate}

\hypertarget{an-example-paper-in-another-folder}{%
\subsection{An example paper in another folder}\label{an-example-paper-in-another-folder}}

Take this simple example (files for this are in \href{https://github.com/ulyngs/oxforddown-external-article}{this GitHub repository}):

\begin{Shaded}
\begin{Highlighting}[]
\NormalTok{|{-}{-}paper\_to\_include}
\NormalTok{|  |{-}{-}my\_paper.Rmd}
\NormalTok{|  |{-}{-}data}
\NormalTok{|  |  |{-}{-}cat\_salt.csv}
\NormalTok{|  |{-}{-}figures}
\NormalTok{|  |  |{-}{-}cat.jpg}
\NormalTok{|}
\NormalTok{|{-}{-}thesis}
\end{Highlighting}
\end{Shaded}

As the chart suggests, you have another folder, \textbf{paper\_to\_include/} living in the same containing folder as your thesis folder.
In the \textbf{paper\_to\_include} folder, the file \textbf{my\_paper.Rmd} is where you write the paper.
In \textbf{my\_paper.Rmd}, you read in a CSV file found in the subfolder \textbf{data/cats.csv}, and also an image from the subfolder \textbf{figures/cat.jpg}.

\hypertarget{step-1-include-paper-as-a-child-document}{%
\subsection{Step 1: Include paper as a child document}\label{step-1-include-paper-as-a-child-document}}

In your thesis folder, create an Rmd file for the chapter where you want to include another paper.
Add one or more code chunks that include R Markdown files from that paper as child documents:

\begin{Shaded}
\begin{Highlighting}[]
\FunctionTok{\# Including an external chapter }

\InformationTok{\textasciigrave{}\textasciigrave{}\textasciigrave{}\{r child = "../paper\_to\_include/my\_paper.Rmd"\}}
\InformationTok{\textasciigrave{}\textasciigrave{}\textasciigrave{}}
\end{Highlighting}
\end{Shaded}

\hypertarget{step-2-make-file-paths-compatible}{%
\subsection{Step 2: Make file paths compatible}\label{step-2-make-file-paths-compatible}}

Use \href{https://rmarkdown.rstudio.com/lesson-6.html}{parameters} to adjust the file path of images based on values you set in the YAML header of an R Markdown file.
In \textbf{my\_paper.Rmd}, create a parameter called \texttt{other\_path} and set it to an empty string:

\begin{Shaded}
\begin{Highlighting}[]
\PreprocessorTok{{-}{-}{-}}
\FunctionTok{title}\KeywordTok{:}\AttributeTok{ }\StringTok{"A fabulous article in a different folder"}
\FunctionTok{params}\KeywordTok{:}
\AttributeTok{  }\FunctionTok{other\_path}\KeywordTok{:}\AttributeTok{ }\StringTok{""}
\PreprocessorTok{{-}{-}{-}}
\end{Highlighting}
\end{Shaded}

In \textbf{my\_paper.Rmd}, put this at the start of the filepath when you read in data or include images:

\begin{Shaded}
\begin{Highlighting}[]
\FunctionTok{library}\NormalTok{(tidyverse)}
\FunctionTok{library}\NormalTok{(knitr)}

\NormalTok{cat\_data }\OtherTok{\textless{}{-}} \FunctionTok{read\_csv}\NormalTok{(}\FunctionTok{str\_c}\NormalTok{(params}\SpecialCharTok{$}\NormalTok{other\_path, }\StringTok{"data/cats.csv"}\NormalTok{))}
\FunctionTok{include\_graphics}\NormalTok{(}\FunctionTok{str\_c}\NormalTok{(params}\SpecialCharTok{$}\NormalTok{other\_path, }\StringTok{"figures/cat.jpg"}\NormalTok{))}
\end{Highlighting}
\end{Shaded}

Finally, in your thesis folder's \textbf{index.Rmd} file, also create the parameter \texttt{other\_path}.
But here, set it to where the \textbf{paper\_to\_include/} folder is relative to your thesis folder:

\begin{Shaded}
\begin{Highlighting}[]
\FunctionTok{params}\KeywordTok{:}
\AttributeTok{  }\FunctionTok{other\_path}\KeywordTok{:}\AttributeTok{ }\StringTok{"../paper\_to\_include/"}
\end{Highlighting}
\end{Shaded}

\hypertarget{note-on-html-output}{%
\subsubsection{Note on HTML output}\label{note-on-html-output}}

Note that if you want to host an HTML version on your thesis online, you will need to include graphics in the content that you host online - the internet obviously won't be able to see filepaths that are just referring to stuff in another folder on your computer!

\hypertarget{step-3-make-sure-header-levels-are-correct}{%
\subsection{Step 3: Make sure header levels are correct}\label{step-3-make-sure-header-levels-are-correct}}

Unless the paper you want to include is also written as a book, your header levels are probably going to be off.
That is, the level 1 headers (\# Some header) you use for main sections in the other paper turns into chaper titles when included in your thesis.

To avoid this, first \emph{increment all heading levels by one in \textbf{paper\_to\_include/my\_paper.Rmd}} (\# Some header -\textgreater{} \#\# Some header).
Then in \textbf{paper\_to\_include/} create a \href{https://bookdown.org/yihui/rmarkdown-cookbook/lua-filters.html\#lua-filters}{lua filter} that decrements header levels by one: Create a text file, save it as \textbf{reduce\_header\_level.lua}, and give it the content below.

\begin{Shaded}
\begin{Highlighting}[]
\KeywordTok{function}\NormalTok{ Header}\OperatorTok{(}\NormalTok{el}\OperatorTok{)}
  \ControlFlowTok{if} \OperatorTok{(}\NormalTok{el}\OperatorTok{.}\NormalTok{level }\OperatorTok{\textless{}=} \DecValTok{1}\OperatorTok{)} \ControlFlowTok{then}
    \FunctionTok{error}\OperatorTok{(}\StringTok{"I don\textquotesingle{}t know how to decrease the level of h1"}\OperatorTok{)}
  \ControlFlowTok{end}
\NormalTok{  el}\OperatorTok{.}\NormalTok{level }\OperatorTok{=}\NormalTok{ el}\OperatorTok{.}\NormalTok{level }\OperatorTok{{-}} \DecValTok{1}
  \ControlFlowTok{return}\NormalTok{ el}
\KeywordTok{end}
\end{Highlighting}
\end{Shaded}

In the YAML header of \textbf{paper\_to\_include/my\_paper.Rmd}, use this filter:

\begin{Shaded}
\begin{Highlighting}[]
\PreprocessorTok{{-}{-}{-}}
\FunctionTok{title}\KeywordTok{:}\AttributeTok{ }\StringTok{"A fabulous article in a different folder"}
\FunctionTok{params}\KeywordTok{:}
\AttributeTok{  }\FunctionTok{other\_path}\KeywordTok{:}\AttributeTok{ }\StringTok{""}
\FunctionTok{output}\KeywordTok{:}
\AttributeTok{  }\FunctionTok{pdf\_document}\KeywordTok{:}\AttributeTok{ }
\AttributeTok{    }\FunctionTok{pandoc\_args}\KeywordTok{:}\AttributeTok{ }\KeywordTok{[}\StringTok{"{-}{-}lua{-}filter=reduce\_header\_level.lua"}\KeywordTok{]}
\PreprocessorTok{{-}{-}{-}}
\end{Highlighting}
\end{Shaded}

Now, your header levels will be correct both when you knit the paper on its own and when its included in your thesis.

\hypertarget{step-4.-make-sure-figure-widths-are-correct}{%
\subsection{Step 4. Make sure figure widths are correct}\label{step-4.-make-sure-figure-widths-are-correct}}

It might be that your figure widths when knitting your paper on its own, and when including it in your thesis, need to be different.
You can again use parameters to set figure widths.

Imagine you want figure width to be 80\% of the page width when knitting your paper on its own, but 100\% in your thesis.
In \textbf{paper\_to\_include/my\_paper.Rmd}, first add a parameter we could call \texttt{out\_width} and set it to the string ``80\%'':

\begin{Shaded}
\begin{Highlighting}[]
\PreprocessorTok{{-}{-}{-}}
\FunctionTok{title}\KeywordTok{:}\AttributeTok{ }\StringTok{"A fabulous article in a different folder"}
\FunctionTok{params}\KeywordTok{:}
\AttributeTok{  }\FunctionTok{other\_path}\KeywordTok{:}\AttributeTok{ }\StringTok{""}
\AttributeTok{  }\FunctionTok{out\_width}\KeywordTok{:}\AttributeTok{ }\StringTok{"80\%"}
\FunctionTok{output}\KeywordTok{:}
\AttributeTok{  }\FunctionTok{pdf\_document}\KeywordTok{:}\AttributeTok{ }
\AttributeTok{    }\FunctionTok{pandoc\_args}\KeywordTok{:}\AttributeTok{ }\KeywordTok{[}\StringTok{"{-}{-}lua{-}filter=reduce\_header\_level.lua"}\KeywordTok{]}
\PreprocessorTok{{-}{-}{-}}
\end{Highlighting}
\end{Shaded}

Then, make sure use that parameter to set the output width when you include figures in \textbf{paper\_to\_include/my\_paper.Rmd}:

\begin{Shaded}
\begin{Highlighting}[]
\InformationTok{\textasciigrave{}\textasciigrave{}\textasciigrave{}\{r, out.width=params$out\_width, fig.cap="A very funny cat"\}}
\InformationTok{include\_graphics(str\_c(params$other\_path, "figures/cat.jpg"))}
\InformationTok{\textasciigrave{}\textasciigrave{}\textasciigrave{}}
\end{Highlighting}
\end{Shaded}

Finally, create the parameter \texttt{out\_width} in your thesis' \textbf{index.Rmd} file:

\begin{Shaded}
\begin{Highlighting}[]
\FunctionTok{params}\KeywordTok{:}
\AttributeTok{  }\FunctionTok{other\_path}\KeywordTok{:}\AttributeTok{ }\StringTok{"../paper\_to\_include/"}
\AttributeTok{  }\FunctionTok{out\_width}\KeywordTok{:}\AttributeTok{ }\StringTok{"80\%"}
\end{Highlighting}
\end{Shaded}

Now, the output width of your figure will be 80\% when knitting your paper on its own, and 100\% when knitting it as child document of your thesis.

\hypertarget{customizing-referencing}{%
\section{Customizing referencing}\label{customizing-referencing}}

\hypertarget{using-a-.csl-file-with-pandoc-instead-of-biblatex}{%
\subsection{Using a .csl file with pandoc instead of biblatex}\label{using-a-.csl-file-with-pandoc-instead-of-biblatex}}

The \texttt{oxforddown} package uses biblatex in LaTeX for referencing.
It is also possible to use pandoc for referencing by providing a .csl file in the YAML header of \textbf{index.Rmd} (likely requiring commenting out the biblatex code in \textbf{templates/template.tex}).
This may be helpful for those who have a .csl file describing the referencing format for a particular journal.
However, note that this approach does not support chapter bibliographies (see Section \ref{biblatex-custom}).

\begin{Shaded}
\begin{Highlighting}[]
\FunctionTok{csl}\KeywordTok{:}\AttributeTok{ ecology.csl}
\end{Highlighting}
\end{Shaded}

\hypertarget{biblatex-custom}{%
\subsection{Customizing biblatex and adding chapter bibliographies}\label{biblatex-custom}}

This section provides one example of customizing biblatex. Much of this code was combined from searches on Stack Exchange and other sources (e.g.~\href{https://tex.stackexchange.com/questions/10682/suppress-in-biblatex}{here}).

In \textbf{templates/template.tex}, one can replace the existing biblatex calls with the following to achieve referencing that looks like this:

(Charmantier and Gienapp 2014)

Charmantier, A. and P. Gienapp (2014). Climate change and timing of avian breeding and migration: evolutionary versus plastic changes. Evolutionary Applications 7(1):15--28. doi: 10.1111/eva.12126.

\begin{Shaded}
\begin{Highlighting}[]
\BuiltInTok{\textbackslash{}usepackage}\NormalTok{[backend=biber,}
\NormalTok{    bibencoding=utf8,}
\NormalTok{    refsection=chapter, }\CommentTok{\% referencing by chapter}
\NormalTok{    style=authoryear, }
\NormalTok{    firstinits=true,}
\NormalTok{    isbn=false,}
\NormalTok{    doi=true,}
\NormalTok{    url=false,}
\NormalTok{    eprint=false,}
\NormalTok{    related=false,}
\NormalTok{    dashed=false,}
\NormalTok{    clearlang=true,}
\NormalTok{    maxcitenames=2,}
\NormalTok{    mincitenames=1,}
\NormalTok{    maxbibnames=10,}
\NormalTok{    abbreviate=false,}
\NormalTok{    minbibnames=3,}
\NormalTok{    uniquelist=minyear,}
\NormalTok{    sortcites=true,}
\NormalTok{    date=year}
\NormalTok{]\{}\ExtensionTok{biblatex}\NormalTok{\}}
\FunctionTok{\textbackslash{}AtEveryBibitem}\NormalTok{\{}\CommentTok{\%}
  \FunctionTok{\textbackslash{}clearlist}\NormalTok{\{language\}}\CommentTok{\%}
  \FunctionTok{\textbackslash{}clearfield}\NormalTok{\{note\}}
\NormalTok{\}}

\FunctionTok{\textbackslash{}DeclareFieldFormat}\NormalTok{\{titlecase\}\{}\FunctionTok{\textbackslash{}MakeTitleCase}\NormalTok{\{\#1\}\}}

\FunctionTok{\textbackslash{}newrobustcmd}\NormalTok{\{}\FunctionTok{\textbackslash{}MakeTitleCase}\NormalTok{\}[1]\{}\CommentTok{\%}
  \FunctionTok{\textbackslash{}ifthenelse}\NormalTok{\{}\FunctionTok{\textbackslash{}ifcurrentfield}\NormalTok{\{booktitle\}}\FunctionTok{\textbackslash{}OR\textbackslash{}ifcurrentfield}\NormalTok{\{booksubtitle\}}\CommentTok{\%}
    \FunctionTok{\textbackslash{}OR\textbackslash{}ifcurrentfield}\NormalTok{\{maintitle\}}\FunctionTok{\textbackslash{}OR\textbackslash{}ifcurrentfield}\NormalTok{\{mainsubtitle\}}\CommentTok{\%}
    \FunctionTok{\textbackslash{}OR\textbackslash{}ifcurrentfield}\NormalTok{\{journaltitle\}}\FunctionTok{\textbackslash{}OR\textbackslash{}ifcurrentfield}\NormalTok{\{journalsubtitle\}}\CommentTok{\%}
    \FunctionTok{\textbackslash{}OR\textbackslash{}ifcurrentfield}\NormalTok{\{issuetitle\}}\FunctionTok{\textbackslash{}OR\textbackslash{}ifcurrentfield}\NormalTok{\{issuesubtitle\}}\CommentTok{\%}
    \FunctionTok{\textbackslash{}OR\textbackslash{}ifentrytype}\NormalTok{\{book\}}\FunctionTok{\textbackslash{}OR\textbackslash{}ifentrytype}\NormalTok{\{mvbook\}}\FunctionTok{\textbackslash{}OR\textbackslash{}ifentrytype}\NormalTok{\{bookinbook\}}\CommentTok{\%}
    \FunctionTok{\textbackslash{}OR\textbackslash{}ifentrytype}\NormalTok{\{booklet\}}\FunctionTok{\textbackslash{}OR\textbackslash{}ifentrytype}\NormalTok{\{suppbook\}}\CommentTok{\%}
    \FunctionTok{\textbackslash{}OR\textbackslash{}ifentrytype}\NormalTok{\{collection\}}\FunctionTok{\textbackslash{}OR\textbackslash{}ifentrytype}\NormalTok{\{mvcollection\}}\CommentTok{\%}
    \FunctionTok{\textbackslash{}OR\textbackslash{}ifentrytype}\NormalTok{\{suppcollection\}}\FunctionTok{\textbackslash{}OR\textbackslash{}ifentrytype}\NormalTok{\{manual\}}\CommentTok{\%}
    \FunctionTok{\textbackslash{}OR\textbackslash{}ifentrytype}\NormalTok{\{periodical\}}\FunctionTok{\textbackslash{}OR\textbackslash{}ifentrytype}\NormalTok{\{suppperiodical\}}\CommentTok{\%}
    \FunctionTok{\textbackslash{}OR\textbackslash{}ifentrytype}\NormalTok{\{proceedings\}}\FunctionTok{\textbackslash{}OR\textbackslash{}ifentrytype}\NormalTok{\{mvproceedings\}}\CommentTok{\%}
    \FunctionTok{\textbackslash{}OR\textbackslash{}ifentrytype}\NormalTok{\{reference\}}\FunctionTok{\textbackslash{}OR\textbackslash{}ifentrytype}\NormalTok{\{mvreference\}}\CommentTok{\%}
    \FunctionTok{\textbackslash{}OR\textbackslash{}ifentrytype}\NormalTok{\{report\}}\FunctionTok{\textbackslash{}OR\textbackslash{}ifentrytype}\NormalTok{\{thesis\}\}}
\NormalTok{    \{\#1\}}
\NormalTok{    \{}\FunctionTok{\textbackslash{}MakeSentenceCase}\NormalTok{\{\#1\}\}\}}
    
\CommentTok{\% \textbackslash{}renewbibmacro\{in:\}\{\}}
\CommentTok{\% suppress "in" for articles}
\CommentTok{\% }
\FunctionTok{\textbackslash{}renewbibmacro}\NormalTok{\{in:\}\{}\CommentTok{\%}
  \FunctionTok{\textbackslash{}ifentrytype}\NormalTok{\{article\}\{\}\{}\FunctionTok{\textbackslash{}printtext}\NormalTok{\{}\FunctionTok{\textbackslash{}bibstring}\NormalTok{\{in\}}\FunctionTok{\textbackslash{}intitlepunct}\NormalTok{\}\}\}}
\CommentTok{\%{-}{-} no "quotes" around titles of chapters/article titles}
\FunctionTok{\textbackslash{}DeclareFieldFormat}\NormalTok{[article, inbook, incollection, inproceedings, misc, thesis, unpublished]}
\NormalTok{\{title\}\{\#1\}}
\CommentTok{\%{-}{-} no punctuation after volume}
\FunctionTok{\textbackslash{}DeclareFieldFormat}\NormalTok{[article]}
\NormalTok{\{volume\}\{\{\#1\}\}}
\CommentTok{\%{-}{-} puts number/issue between brackets}
\FunctionTok{\textbackslash{}DeclareFieldFormat}\NormalTok{[article, inbook, incollection, inproceedings, misc, thesis, unpublished]}
\NormalTok{\{number\}\{}\FunctionTok{\textbackslash{}mkbibparens}\NormalTok{\{\#1\}\} }
\CommentTok{\%{-}{-} and then for articles directly the pages w/o any "pages" or "pp." }
\FunctionTok{\textbackslash{}DeclareFieldFormat}\NormalTok{[article]}
\NormalTok{\{pages\}\{\#1\}}
\CommentTok{\%{-}{-} for some types replace "pages" by "p."}
\FunctionTok{\textbackslash{}DeclareFieldFormat}\NormalTok{[inproceedings, incollection, inbook]}
\NormalTok{\{pages\}\{p. \#1\}}
\CommentTok{\%{-}{-} format 16(4):224{-}{-}225 for articles}
\FunctionTok{\textbackslash{}renewbibmacro*}\NormalTok{\{volume+number+eid\}\{}
  \FunctionTok{\textbackslash{}printfield}\NormalTok{\{volume\}}\CommentTok{\%}
  \FunctionTok{\textbackslash{}printfield}\NormalTok{\{number\}}\CommentTok{\%}
  \FunctionTok{\textbackslash{}printunit}\NormalTok{\{}\FunctionTok{\textbackslash{}addcolon}\NormalTok{\}}
\NormalTok{\}}
\end{Highlighting}
\end{Shaded}

If you would like chapter bibliographies, in addition insert the following code at the end of each chapter, and comment out the entire REFERENCES section at the end of template.tex.

\begin{Shaded}
\begin{Highlighting}[]
\FunctionTok{\textbackslash{}printbibliography}\NormalTok{[segment=}\FunctionTok{\textbackslash{}therefsection}\NormalTok{,heading=subbibliography]}
\end{Highlighting}
\end{Shaded}

\hypertarget{customizing-the-page-headers-and-footers-pdf}{%
\section{Customizing the page headers and footers (PDF)}\label{customizing-the-page-headers-and-footers-pdf}}

This can now be done directly in \textbf{index.Rmd}'s YAML header.
If you are a LaTeX expert and need further customisation that what's currently provided, you can tweak the relevant sections of \textbf{templates/template.tex} - the relevant code is beneath the line that begins \texttt{\textbackslash{}usepackage\{fancyhdr\}}.

\hypertarget{diving-in-to-the-oxthesis-latex-template-pdf}{%
\section{Diving in to the OxThesis LaTeX template (PDF)}\label{diving-in-to-the-oxthesis-latex-template-pdf}}

For LaTeX minded people, you can read through \textbf{templates/template.tex} to see which additional customisation options are available as well as \textbf{templates/ociamthesis.cls} which supplies the base class.
For example, \textbf{template.tex} provides an option for master's degree submissions, which changes identifying information to candidate number and includes a word count.
At the time of writing, you must set this directly in \textbf{template.tex} rather than from the YAML header in \textbf{index.Rmd}.

\hypertarget{customising-to-a-different-university}{%
\section{Customising to a different university}\label{customising-to-a-different-university}}

\hypertarget{the-minimal-route}{%
\subsection{The minimal route}\label{the-minimal-route}}

If the front matter in the OxThesis LaTeX template is suitable to your university, customising \texttt{oxforddown} to your needs could be as simple as putting the name of your institution and the path to your university's logo in \textbf{index.Rmd}:

\begin{Shaded}
\begin{Highlighting}[]
\FunctionTok{university}\KeywordTok{:}\AttributeTok{ University of You}
\FunctionTok{university{-}logo}\KeywordTok{:}\AttributeTok{ figures/your{-}logo{-}here.pdf}
\end{Highlighting}
\end{Shaded}

\hypertarget{replacing-the-entire-title-page-with-your-required-content}{%
\subsection{Replacing the entire title page with your required content}\label{replacing-the-entire-title-page-with-your-required-content}}

If you have a \textbf{.tex} file with some required front matter from your university that you want to replace the OxThesis template's title page altogether, you can provide a filepath to this file in \textbf{index.Rmd}.
\texttt{oxforddown}'s sample content includes and example of this --- if you use the YAML below, your front matter will look like this:

\begin{Shaded}
\begin{Highlighting}[]
\FunctionTok{alternative{-}title{-}page}\KeywordTok{:}\AttributeTok{ front{-}and{-}back{-}matter/alt{-}title{-}page{-}example.tex}
\end{Highlighting}
\end{Shaded}

\noindent
\fbox{\includegraphics[width=0.32\linewidth]{figures/sample-content/alt_frontmatter_example/split/_000001.pdf}} \fbox{\includegraphics[width=0.32\linewidth]{figures/sample-content/alt_frontmatter_example/split/_000002.pdf}} \fbox{\includegraphics[width=0.32\linewidth]{figures/sample-content/alt_frontmatter_example/split/_000003.pdf}} \fbox{\includegraphics[width=0.32\linewidth]{figures/sample-content/alt_frontmatter_example/split/_000004.pdf}} \fbox{\includegraphics[width=0.32\linewidth]{figures/sample-content/alt_frontmatter_example/split/_000005.pdf}} \fbox{\includegraphics[width=0.32\linewidth]{figures/sample-content/alt_frontmatter_example/split/_000006.pdf}}

\hypertarget{troubleshooting}{%
\chapter{Troubleshooting}\label{troubleshooting}}

This chapter describes common errors you may run into, and how to fix them.

\hypertarget{error-failed-to-build-the-bibliography-via-biber}{%
\section{Error: Failed to build the bibliography via biber}\label{error-failed-to-build-the-bibliography-via-biber}}

This can happen if you've had a failed build, perhaps in relation to RStudio shutting down abruptly.

Try doing this:

\begin{enumerate}
\def\labelenumi{\arabic{enumi}.}
\tightlist
\item
  type \texttt{make\ clean-knits} in the terminal tab (or run \texttt{file.remove(list.files(pattern\ =\ "*.(log\textbar{}mtc\textbar{}maf\textbar{}aux\textbar{}bbl\textbar{}blg\textbar{}xml)"))} in the R console) to clean up files generated by LaTeX during a build
\item
  restart your computer
\end{enumerate}

If this does not solve the problem, try using the \href{https://www.overleaf.com/learn/latex/Bibliography_management_with_natbib}{natbib} LaTeX package instead of \href{https://www.overleaf.com/learn/latex/Articles/Getting_started_with_BibLaTeX}{biblatex} for handling references.
To do this, go to \textbf{index.Rmd} and

\begin{enumerate}
\def\labelenumi{\arabic{enumi}.}
\tightlist
\item
  set \texttt{use-biblatex:\ false} and \texttt{use-natbib:\ true}
\item
  set \texttt{citation\_package:\ natbib} under
\end{enumerate}

\begin{Shaded}
\begin{Highlighting}[]
\FunctionTok{output}\KeywordTok{:}
\AttributeTok{  bookdown:}\FunctionTok{:pdf\_book}\KeywordTok{:}
\AttributeTok{    }\FunctionTok{citation\_package}\KeywordTok{:}\AttributeTok{ natbib}
\end{Highlighting}
\end{Shaded}

\begin{savequote}
Alles Gescheite ist schon gedacht worden.\\
Man muss nur versuchen, es noch einmal zu denken.

All intelligent thoughts have already been thought;\\
what is necessary is only to try to think them again.
\qauthor{--- Johann Wolfgang von Goethe \autocite{von_goethe_wilhelm_1829}}\end{savequote}



\hypertarget{conclusion}{%
\chapter*{Conclusion}\label{conclusion}}
\addcontentsline{toc}{chapter}{Conclusion}

If we don't want Conclusion to have a chapter number next to it, we can add the \texttt{\{-\}} attribute.

\hypertarget{more-info}{%
\section*{More info}\label{more-info}}
\addcontentsline{toc}{section}{More info}

And here's some other random info:
the first paragraph after a chapter title or section head \emph{shouldn't be} indented, because indents are to tell the reader that you're starting a new paragraph.
Since that's obvious after a chapter or section title, proper typesetting doesn't add an indent there.

This paragraph, by contrast, \emph{will} be indented as it should because it is not the first one after the `More info' heading.
All hail LaTeX. (If you're reading the HTML version, you won't see any indentation - have a look at the PDF version to understand what in the earth this section is babbling on about).

\startappendices

\hypertarget{the-first-appendix}{%
\chapter{The First Appendix}\label{the-first-appendix}}

This first appendix includes an R chunk that was hidden in the document (using \texttt{echo\ =\ FALSE}) to help with readibility:

\textbf{In 02-rmd-basics-code.Rmd}

\textbf{And here's another one from the same chapter, i.e.~Chapter \ref{code}:}

\hypertarget{the-second-appendix-for-fun}{%
\chapter{The Second Appendix, for Fun}\label{the-second-appendix-for-fun}}


%%%%% REFERENCES
\setlength{\baselineskip}{0pt} % JEM: Single-space References

{\renewcommand*\MakeUppercase[1]{#1}%
\printbibliography[heading=bibintoc,title={\bibtitle}]}


\end{document}
